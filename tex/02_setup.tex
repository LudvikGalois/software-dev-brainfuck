\chapter{Getting set up}

% TODO
{\em Add a short spiel on why we need programming languages}

The programming language we'll be using is called Clojure.  It boasts
a clean and simple syntax, encourages a functional style of
programming but is designed for solving real world problems.

\section{Installing Clojure}

Clojure is not an incredibly popular language, and for most
programming languages, this is a huge problem.  Few people want to
constantly re-invent the wheel, so without a large ecosystem of
libraries it's often hard to justify using a language.  Clojure
bypasses this issue by running on the JVM (Java Virtual Machine) and
having the ability to effortlessly run code written in Java.  This
allows it to leverage the huge Java ecosystem.  This also means that
to run Clojure, you'll need to install a JDK.

% TODO
{\em Add instructions about how to install a JDK here}

\section{Installing Leiningen}

In later chapters, to build Clojure packages, we'll be using a tool
called ``leiningen''.

%TODO
{\em Add link to leiningen}

\section{Choosing an Editor}
For writing code, any text editor will do, but I strongly recommend
you either use emacs with Cider, or vscode with Calva.
