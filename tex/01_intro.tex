\chapter{Introduction}

% TODO
{\em This is just some blathering at this point, feel free to skip}

Programming is a means to an end. The distinction between programming
and software development is often not made, but this is not helpful.
A statisician or scientist writing a program to analyse a dataset is
certainly programming, but it's not software development.  Producing a
maintainable extensible program is rarely the goal for this sort of
program, and the program created is simply part of the data analysis
and not a standalone tool. This book is not about that sort of
programming (although I hope it's still of interest to those who write
such ephemeral programs, especially those who want to turn them into
tools other people can use).

This book will teach you programming, but it won't just cover the
``hard skills'' of what to write to make the computer do what you
need.  It will spend a large amount of time explaining how to write
good code, code that is a pleasure to read and that is easy to extend.
It's also highly opinionated and as such, is only what I consider to
be ``good code'' and not the objective truth of what good code is.

Originally I had planned for this book to use a plethora of different
programming languages, but that placed a large burden on the reader,
especially if this book is also their first proper introduction to
programming.
% TODO clean up following paragraph
In the end, I decided on Clojure. It's not a bastion of purity, but
rather a pragmatic language written to solve real-world problems.  The
syntax is simple, but it's not a toy.

To demonstrate the concepts in this book, we'll use the recurring
theme of trying to produce useful brainfuck programs.  This task may
seem daunting or even outright insane, but I promise that there is a
method to this madness.

%%% Local Variables:
%%% mode: latex
%%% TeX-master: "../book"
%%% End:
