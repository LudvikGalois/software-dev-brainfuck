\chapter{A turing tarpit called Brainfuck}

Clojure has many, many features and functions built-in and we've only
started to scratch the surface.  Brainfuck, on the other hand is
incredibly simple.  It has either 7 or 8 commands depending on how you
want to count them.

\section{An Abstract Machine}

In the last chapter, when we talked about Clojure, we didn't talk
about the {\em abstract machine} it runs on.  Most programmers don't
bother to separate the underlying abstract machine that their
programming language actually talks about from the language itself.
The underlying machine is often as complex as the language and a
formal definition would put many to sleep.  Luckily, since Brainfuck
is so simple, talking about the abstract machine is easy.

There is a row of cells. Each cell can contain a number between 0 and
255 (inclusive).  One of these cells is known as the ``current''
cell.  The leftmost cell starts as the current cell.

In Brainfuck there are 6 actions that can be taken, listed below. Each of these commands is represented by a single character, listed in parentheses.

\begin{enumerate}
\item ({\tt +}) The current cell can be incremented
\item ({\tt -}) The current cell can be decremented
\item ({\tt >}) The cell to the right of the current cell will become the new
  current cell
\item ({\tt <}) The cell to the left of the current cell will become the new
  current cell
\item ({\tt ,}) The value of the current cell is replaced by the next input value
\item ({\tt .}) The value of the current cell is output
\end{enumerate}

The final command(s) are {\tt [} and {\tt ]}.  They always come in
matched pairs.  The meaning of ({\tt [} is ``if the current cell is
not zero, jump to the command after my matching {\tt ]}''. The
meaning of {\tt ]} is ``jump to my matching {\tt [}''.

% TODO
{\em Add a bunch of examples here}

\section{Character Encodings}

{\em Explain why the character 0 is not number 0}

\section{A simple implementation in Clojure}

